% Christian Hubbs
% 28.08.2019

% because I always forget, make sure to run the bibtex command on THIS document!

\documentclass[12pt]{article}
\usepackage[utf8]{inputenc}
\usepackage[english]{babel}
\usepackage{csquotes}
\usepackage{fullpage}
\usepackage{fancyhdr}
\usepackage[pdftex]{graphicx}
\graphicspath{{./images/}}
\usepackage{setspace}
\usepackage{color}
\usepackage{float}
%\hypersetup{draft}
\usepackage[draft]{hyperref}
\usepackage{amsmath}
\usepackage{amsfonts}       % blackboard math symbols
\usepackage{nicefrac}       % compact symbols for 1/2, etc.
\usepackage{microtype}      % microtypography
\usepackage{algorithm}
\usepackage{algpseudocode}
\usepackage{lipsum}
\usepackage{booktabs}
\usepackage[margin=1in]{geometry}
\usepackage{subcaption}
\newcommand{\ra}[1]{\renewcommand{\arraystretch}{#1}}
\usepackage{chngcntr} % Adjusts the equation numbering by section
%\counterwithin*{equation}{section}
\newcommand{\norm}[1]{\left\lVert#1\right\rVert} % Provides norms in mathmode
\newcommand{\citetemp}[1]{(#1)}
\usepackage{wrapfig}
\usepackage{hanging}
\newcommand\tab[1][1cm]{\hspace*{#1}}
\usepackage[toc,page]{appendix}
\usepackage{rotating}
\usepackage{multirow}
\newcommand{\textcite}[1]{\citeauthor{#1}, \citeyear{#1}}
\providecommand{\keywords}[1]{\textit{Keywords:} #1}
%\usepackage{biblatex}
\doublespacing
 
%Import the natbib package and sets a bibliography  and citation styles
\usepackage{natbib}
\bibliographystyle{abbrvnat}

\title{OR-Gym: A Reinforcement Learning Library for Operations Research Problems}
\author{
	Christian D.~Hubbs,\thanks{Department of Chemical Engineering, Carnegie Mellon University, Pittsburgh, PA 15123} \\
	Hector Parra Perez,\footnotemark[1] \\
	Owais Sarwar,\footnotemark[1]\\
	Nikolaos V. Sahinidis,\footnotemark[1] \\
	Ignacio E. Grossmann,\footnotemark[1] \\
	John M. Wassick\thanks{Dow Chemical, Digital Fulfillment Center, Midland, MI 48667}
}

\begin{document}

\maketitle

\begin{abstract}
We introduce OR-Gym, an open-source benchmark in the form of OpenAI Gym for developing reinforcement learning algorithms to address operations research problems.
Reinforcement learning has been widely applied to game-playing and surpassed the best human-level performance in many domains, yet there are few use-cases in industrial or commercial settings.
We apply reinforcement learning to the knapsack, bin packing, travelling salesman, news vendor, max pooling, portfolio optimization, and vehicle routing problems as well as more general, multi-period resource task networks. 
These problems cover logistics, finance, engineering, and are common in many business operation settings.
In each case, we select a prototypical version from the literature to benchmark reinforcement learning and other optimal approaches against. 
\end{abstract}

\keywords{Machine Learning, Reinforcement Learning, Optimization, Scheduling, Stochastic Programming}

\section{Introduction}

\section{Knapsack}

\subsection{Problem Formulation}

\subsection{Benchmark}

\subsection{Reinforcement Learning Algorithm}

\subsection{Results}

\section{Bin Packing}

\subsection{Problem Formulation}

\subsection{Benchmark}

\subsection{Reinforcement Learning Algorithm}

\subsection{Results}

\section{News Vendor}

\subsection{Problem Formulation}

\subsection{Benchmark}

\subsection{Reinforcement Learning Algorithm}

\subsection{Results}

\section{Travelling Salesman}

\subsection{Problem Formulation}

\subsection{Benchmark}

\subsection{Reinforcement Learning Algorithm}

\subsection{Results}

\section{Vehicle Routing}

\subsection{Problem Formulation}

\subsection{Benchmark}

\subsection{Reinforcement Learning Algorithm}

\subsection{Results}

\section{Max Pooling}

\subsection{Problem Formulation}

\subsection{Benchmark}

\subsection{Reinforcement Learning Algorithm}

\subsection{Results}

\section{Portfolio Optimization}

\subsection{Problem Formulation}

\subsection{Benchmark}

\subsection{Reinforcement Learning Algorithm}

\subsection{Results}

\section{Resource Task Network}

\subsection{Problem Formulation}

\subsection{Benchmark}

\subsection{Reinforcement Learning Algorithm}

\subsection{Results}

\section{Conclusion}

\subsection{Future Work}

\newpage

\bibliography{library.bib}

\end{document}